% --------------------------------------
% Document Class
% --------------------------------------
\documentclass[11pt]{article}
% --------------------------------------



% --------------------------------------
% Use Package
% --------------------------------------

% french, english
\usepackage[francais]{babel}

% font, french accent
\usepackage[utf8]{inputenc} 
\usepackage[T1]{fontenc} 

% page layout
\usepackage{geometry}

% hypertext link
\usepackage[pdfpagelabels]{hyperref}

\usepackage{graphicx}
\usepackage{float}
\usepackage{verbatim}
\usepackage{fancyhdr}
\usepackage{amsmath}


% include pdf
\usepackage[final]{pdfpages}


% --------------------------------------



% --------------------------------------
% Page setting
% --------------------------------------
%\pagestyle{empty}
\setlength{\headheight}{15pt}

\setcounter{secnumdepth}{3}
\setcounter{tocdepth}{2}

\makeatletter
\@addtoreset{chapter}{part}
\makeatother 

\hypersetup{         % parametrage des hyperliens
  colorlinks=true,      % colorise les liens
  breaklinks=true,      % permet les retours à la ligne pour les liens trop longs
  urlcolor= blue,       % couleur des hyperliens
  linkcolor= black,     % couleur des liens internes aux documents (index, figures, tableaux, equations,...)
  citecolor= green      % couleur des liens vers les references bibliographiques
}

% --------------------------------------

% --------------------------------------
% Information
% --------------------------------------
\title{Compte-rendu TP3 RdF : Segmentation par binarisation}
\author{Elliot VANEGUE et Gaëtan DEFLANDRE}
% --------------------------------------



% --------------------------------------
% Begin content
% --------------------------------------
\begin{document}

  % Set language to english
  \selectlanguage{francais}

  % Start the page counting
  \pagenumbering{arabic}

  \maketitle
  
  \mbox{}
  \newpage
  \clearpage
  
  \section{Introduction}
  Lors de ce TP, nous allons aborder différentes manières de segmenter une image par 
  classification de ses pixels. Le but de la segmentation concerver les pixels d'un 
  objet et ignorer les pixels du font.
  Nous metterons en avant, plusier méthodes de binarisation, afin de montrer leurs 
  avantages et inconvénients.
  D'abord, le seillage est une binarisation par classification des pixels selon leurs
  niveau de gris.
  Puis, nous verrons la segmentation des pixels en fonction du niveau de texture...
  
  
  \section{Histogramme des niveaux de gris}
  
  \section{Histogramme des niveaux de texture}
  
  \section{Histogramme conjoint}
  
    
  \section{Conclusion}


\end{document}
