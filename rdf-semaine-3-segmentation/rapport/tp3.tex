% --------------------------------------
% Document Class
% --------------------------------------
\documentclass[11pt]{article}
% --------------------------------------



% --------------------------------------
% Use Package
% --------------------------------------

% french, english
\usepackage[francais]{babel}

% font, french accent
\usepackage[utf8]{inputenc} 
\usepackage[T1]{fontenc} 

% page layout
\usepackage{geometry}

% hypertext link
\usepackage[pdfpagelabels]{hyperref}

\usepackage{graphicx}
\usepackage{float}
\usepackage{verbatim}
\usepackage{fancyhdr}
\usepackage{amsmath}


% include pdf
\usepackage[final]{pdfpages}


% --------------------------------------



% --------------------------------------
% Page setting
% --------------------------------------
%\pagestyle{empty}
\setlength{\headheight}{15pt}

\setcounter{secnumdepth}{3}
\setcounter{tocdepth}{2}

\makeatletter
\@addtoreset{chapter}{part}
\makeatother 

\hypersetup{         % parametrage des hyperliens
  colorlinks=true,      % colorise les liens
  breaklinks=true,      % permet les retours à la ligne pour les liens trop longs
  urlcolor= blue,       % couleur des hyperliens
  linkcolor= black,     % couleur des liens internes aux documents (index, figures, tableaux, equations,...)
  citecolor= green      % couleur des liens vers les references bibliographiques
}

% --------------------------------------

% --------------------------------------
% Information
% --------------------------------------
\title{Compte-rendu TP3 RdF : Segmentation par binarisation}
\author{Elliot VANEGUE et Gaëtan DEFLANDRE}
% --------------------------------------



% --------------------------------------
% Begin content
% --------------------------------------
\begin{document}

  % Set language to english
  \selectlanguage{francais}

  % Start the page counting
  \pagenumbering{arabic}

  \maketitle
  
  \mbox{}
  \newpage
  \clearpage
  
  \section*{Introduction}
  Lors de ce TP, nous allons aborder différentes manières de segmenter une image par 
  classification de ses pixels. Le but de la segmentation est de  conserver les pixels d'un 
  objet et ignorer les pixels du fond. Nous mettrons en avant, plusieurs méthodes permettant 
  de déterminer un seuil de binarisation. Nous verrons ainsi leurs avantages et inconvénients.\\
  
  D'abord, nous chercherons le meilleur seuil à partir de l'histogramme de niveaux de gris.
  Ensuite, nous fixerons ce seuil à l'aide de l'histogramme de niveaux de texture.
  Enfin, nous combinerons les deux techniques afin d'obtenir une méthode encore plus 
  fiable pour la segmentation.\\
  
  
  \section{Histogramme des niveaux de gris}
  
   \begin{center}
    \begin{tabular}{|c|c|c|c|c|c|}
      \hline
      \textbf{Nom de l'image} & \textbf{Image source)} & \textbf{Histogramme des gris} & \textbf{Seuil} & \textbf{Image binarisée} & \textbf{Taux d'erreurs en \%}\\
      \hline
      rdf-2-classes-texture-0.png & \includegraphics[width=3cm]{grille2.png} & \includegraphics[width=4cm]{../grille2.png}\\
      \hline
      rdf-2-classes-texture-1.png & \includegraphics[width=3cm]{../grille2.png} & \includegraphics[width=4cm]{../grille5.png}\\
      \hline
      rdf-2-classes-texture-2.png & 9.9139412 & \includegraphics[width=4cm]{../grille8.png}\\
      \hline
      rdf-2-classes-texture-3.png & 5.1679987 & \includegraphics[width=4cm]{../grille9.png}\\
      \hline
      rdf-2-classes-texture-4.png & 2.8661993 & \includegraphics[width=4cm]{../grille10.png}\\
      \hline
    \end{tabular}
  \end{center}
  
  \section{Histogramme des niveaux de texture}
  
  \section{Histogramme conjoint}
  
    
  \section*{Conclusion}


\end{document}
