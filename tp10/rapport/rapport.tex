  % --------------------------------------
% Document Class
% --------------------------------------
\documentclass[a4paper,11pt]{article}
% --------------------------------------



% --------------------------------------
% Use Package
% --------------------------------------


\usepackage[francais]{babel}
%\usepackage{ucs}
\usepackage[utf8]{inputenc}
\usepackage[T1]{fontenc}

\usepackage{makeidx}
\usepackage{color}
\usepackage{graphicx}
\usepackage{float}
\usepackage[hidelinks]{hyperref} 
\usepackage{geometry}
%\usepackage{lastpage}
%\usepackage{marginnote}
\usepackage{fancyhdr}
%\usepackage{titlesec}
%\usepackage{framed}
\usepackage{amsmath}
\usepackage{empheq}
\usepackage{array}
\usepackage{multicol}
\usepackage{csquotes}
%\usepackage{adjustbox}

% insert code
\usepackage{listings}

% define our color
\usepackage{xcolor}

% code color
\definecolor{ligthyellow}{RGB}{250,247,220}
\definecolor{darkblue}{RGB}{5,10,85}
\definecolor{ligthblue}{RGB}{1,147,128}
\definecolor{darkgreen}{RGB}{8,120,51}
\definecolor{darkred}{RGB}{160,0,0}

% other color
\definecolor{ivi}{RGB}{141,107,185}


\lstset{
  language=R,
  captionpos=b,
  extendedchars=true,
  frame=lines,
  numbers=left,
  numberstyle=\tiny,
  numbersep=5pt,
  keepspaces=true,
  breaklines=true,
  showspaces=false,
  showstringspaces=false,
  breakatwhitespace=false,
  stepnumber=1,
  showtabs=false,
  tabsize=3,
  basicstyle=\small\ttfamily,
  backgroundcolor=\color{ligthyellow},
  keywordstyle=\color{ligthblue},
  morekeywords={include, printf, uchar},
  identifierstyle=\color{darkblue},
  commentstyle=\color{darkgreen},
  stringstyle=\color{darkred},
}


% --------------------------------------



% --------------------------------------
% Page setting
% --------------------------------------
%\pagestyle{empty}
\setlength{\headheight}{15pt}

\setcounter{secnumdepth}{3}
\setcounter{tocdepth}{2}

\makeatletter
\@addtoreset{chapter}{part}
\makeatother 

\hypersetup{         % parametrage des hyperliens
  colorlinks=true,      % colorise les liens
  breaklinks=true,      % permet les retours à la ligne pour les liens trop longs
  urlcolor= blue,       % couleur des hyperliens
  linkcolor= black,     % couleur des liens internes aux documents (index, figures, tableaux, equations,...)
  citecolor= green      % couleur des liens vers les references bibliographiques
}

% --------------------------------------

% --------------------------------------
% Information
% --------------------------------------
\title{Compte-rendu TP10 Rdf : Arbres de décision et reconnaissance de visages}
\author{Elliot VANEGUE et Gaëtan DEFLANDRE}
% --------------------------------------

\definecolor{myColor}{rgb}{0.5, 0.1, 0.75}

% --------------------------------------
% Begin content
% --------------------------------------
\begin{document}
  
  % Set language to english
  \selectlanguage{francais}
  
  % Start the page counting
  \pagenumbering{arabic}
  
  \maketitle
  
  \mbox{}
  \newpage
  \clearpage
  
  \section*{Introduction}
  Durant ce TP, nous avons cherché à reconnaître des chiffres grâce à l'utilisation d'arbre
  de décision. La différence avec le TP que nous avons eu précédemment est que nous avons 
  plusieurs image du même chiffre, ce qui implique une gestion de classe. Chaque classe étant
  un chiffre. Nous allons devoir contruire un arbre sur la base d'image d'apprentissage et tester
  cet arbre avec des images de test afin de voir si cet algorithme reconnaît facilement les chiffres.
  
  \section{Préparation des données}
  Pour ce TP, nous travaillons avec une base de chiffre stocké dans plusieurs images. Nous avons dans un premier temps
  du décomposer ces image en plusieurs matrice représentant chaque chiffre avec leur différent exemplaire. Pour cela, nous avons utilisé les 
  fonctions suivante : 
  
  \begin{lstlisting}[caption=Fonction de décomposition de la base]
    splitImageArray <- function( images, nbImage, row, col, nbImage ) {
      result <- array( , dim = c( row, col, nbImage ) );
      for ( i in 1:20 ) {
        for ( j in 1:20 ) {
          result[ , , (j-1)*20+i ] <- images[((i-1)*row+1):(i*row), ((j-1)*col+1):(j*col)];
        }
      }
      return (result);
    }
  \end{lstlisting}
  
  Caractéristique du tableau dans l'exemple des chiffres :
  \begin{itemize}
   \item première et deuxième dimension : taille d'une image (16x16)
   \item troisième dimension : nombre total d'image (11000)
  \end{itemize}

  \section{Méthodologie}
  Afin de pouvoir calculer l'entropie, nous avons besoin dans un premier temps de calculer la probabilité qu'un 
  pixel soit présent dans une classe. Pour cela nous fesons l'adition de toutes les images d'une classe,
  ce qui revient à aditionner la valeur de chaque exemplaire d'un pixel.
  
  \begin{lstlisting}[caption=Fonction de calcul de la probabilité de la présence d'un pixel]
    probabilityPixelOfClassInNApp <- function( data3D, labels, nnumber, napp, mask ) {

      # width
      W <- 16
      # heigth
      H <- 16

      pc <- array(, dim=c(W, H, 0) );

      # pour chaque classe
      for ( c in 0:9 ) {
        csum <- matrix(rep(0, W*H), nrow=H,ncol=W, byrow=TRUE)
        cpt <- 0

        # pour chaque image de la classe, dans les images d'apprentissage 
        for ( i in 1:napp ){
            imgID <- (c * nnumber) + i
            if (mask[imgID]){
                csum <- csum + data3D[,,imgID]
                cpt <- cpt + 1
            }
        }

        csum <- csum / cpt
        pc <- abind( pc, csum, along=3 )
      }

      return ( pc )
   }
  \end{lstlisting}
  
  Le paramètre \enquote{mask} est un tableau qui permet de déterminer si l'image traité dans la boucle doit être 
  prix en compte pour le calcul ou non. Cela permet de différencier les images d'apprentissage des images de test.
  Cette probabilité va nous permettre de calculer l'entropie comme nous l'avons fait dans le TP précédent mais sur chacune
  des classes et non pas sur chacun des éléments (exemplaire de chiffre).
  
  \begin{lstlisting}[caption=Fonction de calcul de l'entropie]
   computeEntropy <- function( pc ) {
 
      # width
      W <- 16
      # heigth
      H <- 16
      sum <- matrix(rep(0, W*H), nrow=H,ncol=W, byrow=TRUE)
      for ( c in 0:9 ){
        sum <- sum + ( log2(pc[,,c+1]^pc[,,c+1]) )
      }
      return ( - sum )
   }
  \end{lstlisting}
  
  \section{Sélection de pixel}
  Pour sélectionner le pixel qui sera le noeud permettant de partager l'ensemble en deux, nous prenons
  l'entropie la plus élevé. Nous pouvons alors chercher dans quel ensemble l'image se trouve en regardant si
  elle possède le pixel sélectionné ou non. Ainsi nous éliminons toutes les images qui n'ont pas le même pixel
  que le chiffre recherché. Nous effectuons ce calcule jusqu'à ce qu'il n'y est plus qu'une seul classe parmis les 
  images.
  
  \begin{lstlisting}[caption=Algorithme principal du TP]
   # nombre d'image par classe
   nnumber <- 1100

   # largeur et hauteur d'une image d'un chiffre
   width <- 16
   heigth <- 16

   # lecture des images binaires
   output <- readUSPSdata("usps")

   data <- output[[1]]
   labels <- output[[2]]

   # la moitié des images sont d'apprentissage
   napp <- 1100/2

   # >>>>>>>>>>>>>>>>>>>>>>>>>>>>>>>>>>>>>>>>>>>>>>>>>>>>>>>>>>>
   # IMAGE TESTE
   # attention il faut prendre une image dont l'indice n'est pas dans napp
   # Tous les 1100 images on passe a une autre classe (autre chiffre),
   # Les images de 1 à 550 sont d'apprentissage.
   #
   # Par exemple de 1 à 550 c'est l'apprentissage des images de 0 et 
   # de 4401 à 4950 c'est l'apprentissage des images de 4
   #
   # ----------------------

   #imgtest <- data[,,850] # 0
   #imgtest <- data[,,1700] # 1
   #imgtest <- data[,,2769] # 2
   #imgtest <- data[,,3820] # 3
   #imgtest <- data[,,4777] # 4
   #imgtest <- data[,,10996] # 9

   # <<<<<<<<<<<<<<<<<<<<<<<<<<<<<<<<<<<<<<<<<<<<<<<<<<<<<<<<<

   # image considere pour le calcul de l'entropie
   nappMask <- getNAppMask(nnumber, napp)
   curentMask <- nappMask

   # /!\ pour le moment c'est du bricolage
   # a mi-chemin entre respect du cours et fait maison

   repeat {

     pc <- probabilityPixelOfClassInNApp(data, labels, nnumber, napp, curentMask)

     entropy <- computeEntropy(pc)

     # condition d'arret
     if ( is.nan(sum(entropy)) ){
        # si il y a des nan dans la matrix entropy on s'arrete    
        break
     }

     bestPixel <- which.max(entropy)

     best.row <- bestPixel %% 16 
     if( best.row==0 ){
        best.row <- 16
     }

     # on arroundit au supérieur car les indices commence à 1
     best.col <- ceiling( bestPixel/16 )

     imagesWithBest <- whichImgContain(best.row, best.col, data, nnumber)

     if(imgtest[best.row,best.col] == 1) {
        curentMask <- imagesWithBest & curentMask
     } else {
        curentMask <- !imagesWithBest & curentMask
     }

     print(paste("Nombre de pixels restant", (sum(curentMask))))

  }

  proba <- getProbaComeFromClass(curentMask, napp)
  # on fait -1 car le chiffre 0 commence a l'indice 1
  print(paste("L'image testé appartient probablement à la classe", (which.max(proba)-1)))
  \end{lstlisting}
  
  \section*{Conclusion}
  Nous avons pu voir que cette méthode de reconnaissance de forme ne permet pas d'avoir un résultat
  très fiable. En effet plusieurs chiffres, sont mal reconnu par l'algorithme étudié lors de ce TP.
  La valeur des pixels sur une image en dégradé de gris ne suffit donc pas, il faudrait utiliser 
  d'avantage de paramètre comme la couleur de la peau ou la position de certaines caractèristique du visage
  pour rendre cette algorithme plus fiable.
  
\end{document}  