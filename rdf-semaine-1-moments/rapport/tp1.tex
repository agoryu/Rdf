% --------------------------------------
% Document Class
% --------------------------------------
\documentclass{article}
% --------------------------------------



% --------------------------------------
% Use Package
% --------------------------------------

% french, english
\usepackage[francais]{babel}

% font, french accent
\usepackage[utf8]{inputenc} 
\usepackage[T1]{fontenc} 

% page layout
\usepackage{geometry}

% hypertext link
\usepackage[pdfpagelabels]{hyperref}

\usepackage{graphicx}
\usepackage{float}
\usepackage{verbatim}
\usepackage{fancyhdr}
\usepackage{amsmath}


% include pdf
\usepackage[final]{pdfpages}


% --------------------------------------



% --------------------------------------
% Page setting
% --------------------------------------
%\pagestyle{empty}
\setlength{\headheight}{15pt}

\setcounter{secnumdepth}{3}
\setcounter{tocdepth}{2}

\makeatletter
\@addtoreset{chapter}{part}
\makeatother 

\hypersetup{         % parametrage des hyperliens
  colorlinks=true,      % colorise les liens
  breaklinks=true,      % permet les retours à la ligne pour les liens trop longs
  urlcolor= blue,       % couleur des hyperliens
  linkcolor= black,     % couleur des liens internes aux documents (index, figures, tableaux, equations,...)
  citecolor= green      % couleur des liens vers les references bibliographiques
}

% --------------------------------------

% --------------------------------------
% Information
% --------------------------------------
\title{Compte-rendu TP1 Rdf : Moments d'une forme}
\author{Elliot VANEGUE et Gaëtan DEFLANDRE}
% --------------------------------------



% --------------------------------------
% Begin content
% --------------------------------------
\begin{document}

  % Set language to english
  \selectlanguage{francais}

  % Start the page counting
  \pagenumbering{arabic}

  \maketitle
  
  \mbox{}
  \newpage
  \clearpage
  
  \section{Introduction}
  Lors de ce TP nous allons voir comment exploiter les moments d'une forme. Les moments
  d'une forme sont des valeurs que nous allons exploiter afin de différencier les formes
  de base d'une image.
  
  \section{Moments d'une forme}
  Dans un premier temps, nous allons voir que la matrice d'inertie d'une forme permet de calculer 
  les moments principaux et axe principal d'inertie. Nous allons effectuer les calculs sur des 
  rectangles ayant des positions différentes afin de voir quels impactes cela aura sur les 
  éléments cités précédemment.\\
  
  \begin{center}
    \begin{tabular}{|c|c|c|c|}
      \hline
      \textbf{image} & \textbf{matrice d'inertie} & \textbf{moments principaux} & \textbf{axe principal} \\
      \hline
      rectangle horizontal & $\begin{pmatrix}
			      1360 & 0 \\
			      0 & 80 
			      \end{pmatrix}$ 
			  & 1360; 80
			  &  $\begin{pmatrix}
			      -1 & 0 \\
			      0 & -1 
			      \end{pmatrix}$\\
      \hline
      rectangle vertical & $\begin{pmatrix}
			      80 & 0 \\
			      0 & 1360 
			      \end{pmatrix}$
			  & 1360; 80
			  &  $\begin{pmatrix}
			      0 & -1 \\
			      1 & 0 
			      \end{pmatrix}$\\
      \hline
      rectangle diagonal lisse & $\begin{pmatrix}
			      745 & -647 \\
			      -647 & 748,4 
			      \end{pmatrix}$
			  & 1394,7; 99,7
			  &  $\begin{pmatrix}
			      -0,7 & -0,7 \\
			      0,7 & -0,7 
			      \end{pmatrix}$\\
      \hline
      rectangle diagonal & $\begin{pmatrix}
			      678,5 & -619,5 \\
			      -619,5 & 678,5 
			      \end{pmatrix}$
			  & 1298; 59
			  &  $\begin{pmatrix}
			      -0,7 & -0,7 \\
			      0,7 & -0,7 
			      \end{pmatrix}$\\
      \hline
    \end{tabular}
  \end{center}
  
  Suite aux résultats précédents nous voyons que l'axe d'inertie nous permet de déterminer 
  l'angle de rotation du rectangle afin de savoir si celui-ci est horizontal, vertical ou diagonal. 
  On peut également voir que les moments principaux sont presque identiques quelque soit la rotation 
  du rectangle, et le fait que le rectangle soit lissé ou non, ne change pas les valeurs et vecteurs 
  propres.\\
  
  \begin{center}
    \begin{tabular}{|c|c|}
      \hline
      \textbf{image} & \textbf{moments principaux} \\
      \hline
      carré côté 6 & 105; 105 \\
      \hline
      carré côté 10 & 825; 825 \\
      \hline
      carré rotation 30deg & 843,3; 842,4 \\
      \hline
      carré rotation 45deg & 841,5; 838,5 \\
      \hline
      carré côté 20 & 13 300; 13 300 \\
      \hline
    \end{tabular}
  \end{center}
  
  Avec ces résultats, nous voyons l'inégalité suivante:
  $$
    valeurs\_propres\ carre\_cote\_6 < valeurs\_propres\ carre\_cote\_10 < valeurs\_propres\ carre\_cote\_20
  $$
  
  Nous remarquons, donc, que ces moments principaux ne sont pas invariants en échelle, mais 
  ils le sont en rotation.
  Il faut donc trouver une autre solution pour que le calcul donne un résultat identique
  pour une forme quelque soit son rapport homothétique.
  
  \newpage
  
  \section{Moments normalisés}
  Nous allons utiliser un calcul permettant de normaliser les moments principaux afin d'avoir
  une invariance en rotation et en échelle.\\
  
   \begin{center}
    \begin{tabular}{|c|c|}
      \hline
      \textbf{image} & \textbf{moments principaux} \\
      \hline
      carré côté 6 & 0,081; 0,081 \\
      \hline
      carré côté 10 & 0,083; 0,083 \\
      \hline
      carré rotation 30deg & 0,084; 0,084 \\
      \hline
      carré rotation 45deg & 0,085; 0,085 \\
      \hline
      carré côté 20 & 0,083; 0,083 \\
      \hline
       & \\
      \hline
      rectangle horizontal & 0.332; 0,019 \\
      \hline
      rectangle vertical & 0.332; 0,019 \\
      \hline
      rectangle diagonal lisse & 0.335; 0,023 \\
      \hline
      rectangle diagonal & 0.385; 0.017 \\
      \hline
       & \\
      \hline
      triangle côté 10 & 0.100; 0.095\\
      \hline
      triangle rotation 15deg & 0.100; 0.095\\
      \hline
      triangle rotation 45deg & 0.100; 0.095\\
      \hline
      triangle rotation 60deg & 0.101; 0.093\\
      \hline
    \end{tabular}
  \end{center}
  
  Ces résultats montrent qu'en normalisant les moments principaux, le changement d'échelle les 
  fait nettement moins varier. Donc, nous obtenons une méthode qui donne des résultats invariants 
  en échelle et en rotation.\\
  
   \newpage
  
  \section{Moments invariants}
  Nous allons maintenant calculer les cinq premiers moments de Hu afin de déterminer les atouts
  et les inconvénients de cette méthode.
  
  \begin{center}
    \begin{tabular}{|c|c|c|c|c|c|}
      \hline
      \textbf{image} & \textbf{moment 1} & \textbf{moment 2} & \textbf{moment 3} & \textbf{moment 4} & \textbf{moment 5}\\
      \hline
      carré côté 6 & 0.162037 & 0 & 0 & 0 & 0 \\
      \hline
      carré côté 10 & 0.165 & 0 & 0 & 0 & 0 \\
      \hline
      carré rotation 30deg & 0.1681347 & $7.378984e^{-09}$ & $2.74901e^{-08}$ & $8.00053e^{-10}$ & $-2.026355e^{-18}$ \\
      \hline
      carré rotation 45deg & 0.1705641 & $9.160345e^{-08}$ & $2.829153e^{-08}$ & $4.322032e^{-09}$ & $1.818075e^{-18}$ \\
      \hline
      & & & & & \\
      \hline
      rectangle horizontal & 0.3515625 & 0.09765625 & 0 & 0 & 0 \\
      \hline
      rectangle vertical & 0.3515625 & 0.09765625 & 0 & 0 & 0 \\
      \hline
      rectangle diagonal lisse & 0.3589944 & 0.09676744 & $2.230677e^{-06}$ & $1.792061e^{-06}$ & $3.340324e^{-12}$ \\
      \hline
      rectangle diagonal & 0.4033888 & 0.1356534 & 0 & 0 & 0 \\
      \hline
      & & & & & \\
      \hline
      triangle côté 10 & 0.1956167 & $3.524209e^{-05}$ & 0.004551521 & $4.91483e^{-07}$ & $-1.004697e^{-11}$\\
      \hline
      triangle rotation 15deg & 0.1951716 & $3.03244e^{-05}$ & 0.004574826 & $8.943819e^{-07}$ & $2.253919e^{-11}$\\
      \hline
      triangle rotation 45deg & 0.1956615 & $2.859357e^{-05}$ & 0.004614545 & $1.474173e^{-07}$ & $3.523543e^{-12}$\\
      \hline
      & & & & & \\
      \hline
      0 & 0.4561521 & 0.01753315 & $7.632676e^{-08}$ & $1.623057e^{-07}$ & $-1.747216e^{-14}$\\
      \hline
      1 & 0.6165951 & 0.2460521 & 0.01457684 & 0.002890858 & $1.288171e^{-05}$\\
      \hline
      2 & 0.5967148 & 0.1355688 & 0.007449113 & 0.002414322 & $2.801895e^{-06}$\\
      \hline
      3 & 0.5273825 & 0.08907071 & 0.016224 & 0.004589431 & $-3.616123e^{-06}$\\
      \hline
    \end{tabular}
  \end{center}
  
  %Pour cette méthode, nous pouvons voir qu'il faut surtout prendre en compte les exposants des valeurs.
  %Cette méthode permet de reconnaitre une forme en faisant abstraction de sa taille. Par contre, lorsque 
  %la forme subit une rotation, certaines valeurs sont modifiées. On peut voir ce phénomène sur les rotations 
  %du carré et sur l'image du rectangle diagonal lisse.\\
  
  Avec la méthode de Hu, nous constatons que les moments 1 et 2 sont invariants en translation, en 
  rotation et en chargement d'échelle. Nous voyons quelques petites variances sur l'ensemble des 
  moments liés à l'anticrénelage des images. Ces variances sont négligeables.\\
  
  De plus, on remarque que les exposants restent malgré tout assez proche pour les formes identiques.\\ 
  
  On peut donc dire, que cette méthode ne permet pas de dire que deux formes sont identiques, mais 
  elle permet de savoir si elles sont différentes. Par exemple, certains résultats des chiffres sont 
  proches.
  
  \section{Conclusion}
  Pour conclure, la meilleure méthode de calcul de moments, parmi celles étudiées, pour faire de la 
  reconnaissance  de formes est le calcul de moments normalisés, car celui-ci reconnaît une forme 
  sans prendre en compte sa rotation ou son échelle.
    
\end{document}