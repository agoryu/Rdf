% --------------------------------------
% Document Class
% --------------------------------------
\documentclass{article}
% --------------------------------------



% --------------------------------------
% Use Package
% --------------------------------------

% french, english
\usepackage[francais]{babel}

% font, french accent
\usepackage[utf8]{inputenc} 
\usepackage[T1]{fontenc} 

% page layout
\usepackage{geometry}

% hypertext link
\usepackage[pdfpagelabels]{hyperref}

\usepackage{graphicx}
\usepackage{float}
\usepackage{verbatim}
\usepackage{fancyhdr}
\usepackage{amsmath}


% include pdf
\usepackage[final]{pdfpages}


% --------------------------------------



% --------------------------------------
% Page setting
% --------------------------------------
%\pagestyle{empty}
\setlength{\headheight}{15pt}

\setcounter{secnumdepth}{3}
\setcounter{tocdepth}{2}

\makeatletter
\@addtoreset{chapter}{part}
\makeatother 

\hypersetup{         % parametrage des hyperliens
  colorlinks=true,      % colorise les liens
  breaklinks=true,      % permet les retours à la ligne pour les liens trop longs
  urlcolor= blue,       % couleur des hyperliens
  linkcolor= black,     % couleur des liens internes aux documents (index, figures, tableaux, equations,...)
  citecolor= green      % couleur des liens vers les references bibliographiques
}

% --------------------------------------

% --------------------------------------
% Information
% --------------------------------------
\title{Compte-rendu TP1 Rdf : Moments d'une forme}
\author{Elliot VANEGUE et Gaëtan DEFLANDRE}
% --------------------------------------



% --------------------------------------
% Begin content
% --------------------------------------
\begin{document}

  % Set language to english
  \selectlanguage{francais}

  % Start the page counting
  \pagenumbering{arabic}

  \maketitle
  
  \mbox{}
  \newpage
  \clearpage
  
  \section{Introduction}
  Lors de ce TP nous allons voir comment exploiter les moments d'une forme. Les moments
  d'une forme sont des valeurs que nous allons exploiter afin de différencier les formes
  de base d'une image.
  
  \section{Moments d'une forme}
  Dans un premier temps, nous allons voir que la matrice d'inertie d'une forme permet de calculer 
  les moments principaux et axe principal d'inertie. Nous allons effectuer les calcules sur des 
  rectangles ayant des positions différentes afin de voir quels impactes cela aura sur les 
  élément cité précédement.\\
  
  \begin{center}
    \begin{tabular}{|c|c|c|c|}
      \hline
      \textbf{image} & \textbf{matrice d'inertie} & \textbf{moments principaux} & \textbf{axe principale} \\
      \hline
      rectangle horizontal & $\begin{pmatrix}
			      1360 & 0 \\
			      0 & 80 
			      \end{pmatrix}$ 
			  & 1360; 80
			  &  $\begin{pmatrix}
			      -1 & 0 \\
			      0 & -1 
			      \end{pmatrix}$\\
      \hline
      rectangle vertical & $\begin{pmatrix}
			      80 & 0 \\
			      0 & 1360 
			      \end{pmatrix}$
			  & 1360; 80
			  &  $\begin{pmatrix}
			      0 & -1 \\
			      1 & 0 
			      \end{pmatrix}$\\
      \hline
      rectangle diagonal lisse & $\begin{pmatrix}
			      745 & -647 \\
			      -647 & 748,4 
			      \end{pmatrix}$
			  & 1394,7; 99,7
			  &  $\begin{pmatrix}
			      -0,7 & -0,7 \\
			      0,7 & -0,7 
			      \end{pmatrix}$\\
      \hline
      rectangle diagonal & $\begin{pmatrix}
			      678,5 & -619,5 \\
			      -619,5 & 678,5 
			      \end{pmatrix}$
			  & 1298; 59
			  &  $\begin{pmatrix}
			      -0,7 & -0,7 \\
			      0,7 & -0,7 
			      \end{pmatrix}$\\
      \hline
    \end{tabular}
  \end{center}
  
  Suite aux résultats précédents nous voyons que l'axe d'inertie nous permet de déterminer 
  l'angle de rotation du rectangle afin de savoir si celui-ci est horizontal, vertical ou diagonal. 
  On peut également voir que les moments principaux sont presque identiques quelque soit la rotation du rectangle,
  et le fait que le rectangle soit flouté ou pas ne change rien aux valeurs et vecteurs propre.\\
  
  \begin{center}
    \begin{tabular}{|c|c|}
      \hline
      \textbf{image} & \textbf{moments principaux} \\
      \hline
      carré côté 6 & 105; 105 \\
      \hline
      carré côté 10 & 825; 825 \\
      \hline
      carré rotation 30deg & 843,3; 842,4 \\
      \hline
      carré rotation 45deg & 841,5; 838,5 \\
      \hline
      carré côté 20 & 13 300; 13 300 \\
      \hline
    \end{tabular}
  \end{center}
  
  Avec ces résultats, nous remarquons que les moments principaux prennent en compte
  la taille de la forme présente dans l'image.
  Il faut donc trouver une autre solution pour que le calcul donne un résultat identique
  pour une forme quelle que soit sa taille.
  
  \section{Moments normalisé}
  Nous allons utiliser un calcul permettant de normaliser les moments principaux afin d'avoir
  une diminution de la variation des résultats obtenus sur une même forme.\\
  
   \begin{center}
    \begin{tabular}{|c|c|}
      \hline
      \textbf{image} & \textbf{moments principaux} \\
      \hline
      carré côté 6 & 0,081; 0,081 \\
      \hline
      carré côté 10 & 0,083; 0,083 \\
      \hline
      carré rotation 30deg & 0,084; 0,084 \\
      \hline
      carré rotation 45deg & 0,085; 0,085 \\
      \hline
      carré côté 20 & 0,083; 0,083 \\
      \hline
       & \\
      \hline
      rectangle horizontal & 0.332; 0,019 \\
      \hline
      rectangle vertical & 0.332; 0,019 \\
      \hline
      rectangle diagonal lisse & 0.335; 0,023 \\
      \hline
      rectangle diagonal & 0.385; 0.017 \\
      \hline
       & \\
      \hline
      triangle côté 10 & 0.100; 0.095\\
      \hline
      triangle rotation 15deg & 0.100; 0.095\\
      \hline
      triangle rotation 45deg & 0.100; 0.095\\
      \hline
      triangle rotation 60deg & 0.101; 0.093\\
      \hline
    \end{tabular}
  \end{center}
  
  Les résultats montrent que le changement d'échelle fait nettement moins varier les moments
  principaux. L'ensemble des moments principaux d'une forme pour des tailles et des rotations
  différentes restent quasiment égaux.\\
  
  \section{Moments invariant}
  Nous allons maintenant calculer les cinq premier moments de Hu afin de déterminer les atouts
  et les inconvénient de cette méthode.
  
  \begin{center}
    \begin{tabular}{|c|c|c|c|c|c|}
      \hline
      \textbf{image} & \textbf{moments 1} & \textbf{moments 2} & \textbf{moments 3} & \textbf{moments 4} & \textbf{moments 5}\\
      \hline
      carré côté 6 & 0.162037 & 0 & 0 & 0 & 0 \\
      \hline
      carré côté 10 & 0.165 & 0 & 0 & 0 & 0 \\
      \hline
      carré rotation 30deg & 0.1681347 & $5.59215^{-09}$ & $2.74901^{-08}$ & $8.00053^{-10}$ & $-2.026355^{-18}$ \\
      \hline
      carré rotation 45deg & 0.1705641 & $2.586728^{-08}$ & $2.829153^{-08}$ & $4.322032^{-09}$ & $1.818075^{-18}$ \\
      \hline
      & & & & & \\
      \hline
      rectangle horizontal & 0.3515625 & 0.09765625 & 0 & 0 & 0 \\
      \hline
      rectangle vertical & 0.3515625 & 0.09765625 & 0 & 0 & 0 \\
      \hline
      rectangle diagonal lisse & 0.3589944 & 0.02419236 & $2.230677^{-06}$ & $1.792061^{-06}$ & $3.340324^{-12}$ \\
      \hline
      rectangle diagonal & 0.4033888 & 0.03391335 & 0 & 0 & 0 \\
      \hline
      & & & & & \\
      \hline
      triangle côté 10 & 0.1956167 & $3.103136^{-05}$ & 0.004551521 & $4.91483^{-07}$ & $-1.004697^{-11}$\\
      \hline
      triangle rotation 15deg & 0.1951716 & $1.482205^{-05}$ & 0.004574826 & $8.943819^{-07}$ & $2.253919^{-11}$\\
      \hline
      triangle rotation 45deg & 0.1956615 & $9.161456^{-06}$ & 0.004614545 & $1.474173^{-07}$ & $3.523543^{-12}$\\
      \hline
      & & & & & \\
      \hline
      0 & 0.4561521 & 0.01753315 & $7.632676^{-08}$ & $1.623057^{-07}$ & $-1.747216^{-14}$\\
      \hline
      1 & 0.6165951 & 0.2373792 & 0.01457684 & 0.002890858 & $1.288171^{-05}$\\
      \hline
      2 & 0.5967148 & 0.1315581 & 0.007449113 & 0.002414322 & $2.801895^{-06}$\\
      \hline
      3 & 0.5273825 & 0.08905541 & 0.016224 & 0.004589431 & $-3.616123^{-06}$\\
      \hline
    \end{tabular}
  \end{center}
  
  Pour cette méthode, nous pouvons voir qu'il faut surtout prendre en compte les exposants des valeurs.
  Cette méthode permet de reconnaitre une forme en faisant abstraction de sa taille. Par contre, lorsque 
  la forme subit une rotation, certaines valeurs sont modifiées. On peut voir ce phénomène sur les rotations 
  du carré et sur l'image du rectangle diagonal lisse.\\
  
  De plus, cette méthode ne permet pas de différencier les nombres, car pour bon nombre d'entre eux les moments d'Hu sont quasiment identiques.
  
  \section{Conclusion}
  Pour conclure, la meilleure méthode de calcul de moment, parmi celle étudiée, pour faire de la reconnaissance 
  de formes est le calcul de moment normalisé, car celui-ci reconnaît une forme sans prendre en compte sa rotation ou sa taille.
    
\end{document}