% --------------------------------------
% Document Class
% --------------------------------------
\documentclass[a4paper,11pt]{article}
% --------------------------------------



% --------------------------------------
% Use Package
% --------------------------------------


\usepackage[francais]{babel}
%\usepackage{ucs}
\usepackage[utf8]{inputenc}
\usepackage[T1]{fontenc}

\usepackage{makeidx}
\usepackage{color}
\usepackage{graphicx}
\usepackage{float}
\usepackage[hidelinks]{hyperref} 
\usepackage{geometry}
%\usepackage{lastpage}
%\usepackage{marginnote}
\usepackage{fancyhdr}
%\usepackage{titlesec}
%\usepackage{framed}
\usepackage{amsmath}
\usepackage{empheq}
\usepackage{array}
\usepackage{multicol}
%\usepackage{adjustbox}

% insert code
\usepackage{listings}

% define our color
\usepackage{xcolor}

% code color
\definecolor{ligthyellow}{RGB}{250,247,220}
\definecolor{darkblue}{RGB}{5,10,85}
\definecolor{ligthblue}{RGB}{1,147,128}
\definecolor{darkgreen}{RGB}{8,120,51}
\definecolor{darkred}{RGB}{160,0,0}

% other color
\definecolor{ivi}{RGB}{141,107,185}


\lstset{
    language=Scilab,
    captionpos=b,
    extendedchars=true,
    frame=lines,
    numbers=left,
    numberstyle=\tiny,
    numbersep=5pt,
    keepspaces=true,
    breaklines=true,
    showspaces=false,
    showstringspaces=false,
    breakatwhitespace=false,
    stepnumber=1,
    showtabs=false,
    tabsize=3,
    basicstyle=\small\ttfamily,
    backgroundcolor=\color{ligthyellow},
    keywordstyle=\color{ligthblue},
    morekeywords={include, printf, uchar},
    identifierstyle=\color{darkblue},
    commentstyle=\color{darkgreen},
    stringstyle=\color{darkred},
}


% --------------------------------------



% --------------------------------------
% Page setting
% --------------------------------------
%\pagestyle{empty}
\setlength{\headheight}{15pt}

\setcounter{secnumdepth}{3}
\setcounter{tocdepth}{2}

\makeatletter
\@addtoreset{chapter}{part}
\makeatother 

\hypersetup{         % parametrage des hyperliens
  colorlinks=true,      % colorise les liens
  breaklinks=true,      % permet les retours à la ligne pour les liens trop longs
  urlcolor= blue,       % couleur des hyperliens
  linkcolor= black,     % couleur des liens internes aux documents (index, figures, tableaux, equations,...)
  citecolor= green      % couleur des liens vers les references bibliographiques
}

% --------------------------------------

% --------------------------------------
% Information
% --------------------------------------
\title{Compte-rendu TP8 Rdf : Classifcation non supervisée}
\author{Elliot VANEGUE et Gaëtan DEFLANDRE}
% --------------------------------------

\definecolor{myColor}{rgb}{0.5, 0.1, 0.75}

% --------------------------------------
% Begin content
% --------------------------------------
\begin{document}

% Set language to english
  \selectlanguage{francais}

  % Start the page counting
  \pagenumbering{arabic}

  \maketitle
  
  \mbox{}
  \newpage
  \clearpage
  
  \section*{Introduction}
  Jusque là nous avons vu des méthodes de classification supervisée, c'est à dire qu'elles produisaient
  des règle automatiquement à partir de données fournit au préalable. 
  Lors de ce TP, nous allons voir une méthode non supervisée qui est la méthode du K-means. Cette méthode
  va permettre de séparer les données de l'image en plusieurs groupe sans fournir de base données. Nous allons
  nous servir de cette technique afin de trouver le seuil le plus approprié à la binarisation d'une image.
  
  \section{Classification des données Iris par la méthode K-means}
  Dans un premier temps, nous classifions les données Iris en trois classes représentant les trois espèces
  d'iris. Pour différencier les iris nous avons quatre caractéristiques fournit dans les données : la largeur et
  la longueur du séparle et la largeur et la hauteur du pétale.\\
  
  \begin{figure}[H]
  \center
   \includegraphics[width=9cm]{resultat/separation_espece.png}
   \caption{Graphique de séparation des données des trois espèces d'iris}
  \end{figure}

  Nous appliquons la classification K-means sur ces données afin de voir si celui-ci sépare correctement les données.
  Le principe de l'algorithme du K-means est de minimiser la distance entre le centre d'une classe et les données
  qui la constitue. Pour cela, ces centres sont le plus éloigné entre eux à la première itération, puis ils se 
  déplacent à chaque itération jusqu'à se retrouver au centre des données d'une classe.\\
  
  Lorsque nous effectuons cette algorithme sur les données Iris avec quinze itération, nous pouvons voir que l'un des centres se trouve 
  entre deux classes tandis que les deux autres se retrouvent sur la même classe.
  
  \begin{figure}[H]
  \center
   \includegraphics[width=9cm]{resultat/kmeans.png}
   \caption{Résultat de la classification des données Iris par K-means}
  \end{figure}
  
  %TODO suite de la premiere partie -> reprendre en 1.3
  
  \section{Segmentation d'une image de textures par classification non supervisée des pixels}
  Maintenant nous allons segmenter l'image des gâteaux étudié lors du TP3 grâce à la classification
  K-means. Nous créons une matrice composé des données des deux images. Nous pouvons voir que le nuage
  de point est composé de deux forme distincte : l'une est à l'horizontale et relativement grosse, tandis 
  que l'autre est parfaitement à la verticale.
  
  \begin{figure}[H]
  \center
   \includegraphics[width=9cm]{resultat/image_combine.png}
   \caption{Graphique des données des deux images de gâteaux}
  \end{figure}
  
  Nous effectuons la classification des données précédente avec K-means en 30 itération. Le résultat
  nous donne une classification 
  
\end{document}  