  % --------------------------------------
% Document Class
% --------------------------------------
\documentclass[a4paper,11pt]{article}
% --------------------------------------



% --------------------------------------
% Use Package
% --------------------------------------


\usepackage[francais]{babel}
%\usepackage{ucs}
\usepackage[utf8]{inputenc}
\usepackage[T1]{fontenc}

\usepackage{makeidx}
\usepackage{color}
\usepackage{graphicx}
\usepackage{float}
\usepackage[hidelinks]{hyperref} 
\usepackage{geometry}
%\usepackage{lastpage}
%\usepackage{marginnote}
\usepackage{fancyhdr}
%\usepackage{titlesec}
%\usepackage{framed}
\usepackage{amsmath}
\usepackage{empheq}
\usepackage{array}
\usepackage{multicol}
%\usepackage{adjustbox}

% insert code
\usepackage{listings}

% define our color
\usepackage{xcolor}

% code color
\definecolor{ligthyellow}{RGB}{250,247,220}
\definecolor{darkblue}{RGB}{5,10,85}
\definecolor{ligthblue}{RGB}{1,147,128}
\definecolor{darkgreen}{RGB}{8,120,51}
\definecolor{darkred}{RGB}{160,0,0}

% other color
\definecolor{ivi}{RGB}{141,107,185}


\lstset{
  language=R,
  captionpos=b,
  extendedchars=true,
  frame=lines,
  numbers=left,
  numberstyle=\tiny,
  numbersep=5pt,
  keepspaces=true,
  breaklines=true,
  showspaces=false,
  showstringspaces=false,
  breakatwhitespace=false,
  stepnumber=1,
  showtabs=false,
  tabsize=3,
  basicstyle=\small\ttfamily,
  backgroundcolor=\color{ligthyellow},
  keywordstyle=\color{ligthblue},
  morekeywords={include, printf, uchar},
  identifierstyle=\color{darkblue},
  commentstyle=\color{darkgreen},
  stringstyle=\color{darkred},
}


% --------------------------------------



% --------------------------------------
% Page setting
% --------------------------------------
%\pagestyle{empty}
\setlength{\headheight}{15pt}

\setcounter{secnumdepth}{3}
\setcounter{tocdepth}{2}

\makeatletter
\@addtoreset{chapter}{part}
\makeatother 

\hypersetup{         % parametrage des hyperliens
  colorlinks=true,      % colorise les liens
  breaklinks=true,      % permet les retours à la ligne pour les liens trop longs
  urlcolor= blue,       % couleur des hyperliens
  linkcolor= black,     % couleur des liens internes aux documents (index, figures, tableaux, equations,...)
  citecolor= green      % couleur des liens vers les references bibliographiques
}

% --------------------------------------

% --------------------------------------
% Information
% --------------------------------------
\title{Compte-rendu TP9 Rdf : Arbres de décision}
\author{Elliot VANEGUE et Gaëtan DEFLANDRE}
% --------------------------------------

\definecolor{myColor}{rgb}{0.5, 0.1, 0.75}

% --------------------------------------
% Begin content
% --------------------------------------
\begin{document}
  
  % Set language to english
  \selectlanguage{francais}
  
  % Start the page counting
  \pagenumbering{arabic}
  
  \maketitle
  
  \mbox{}
  \newpage
  \clearpage
  
  \section*{Introduction}
   Durant ce TP, nous allons voir une nouvelle méthode permettant la séparation des données à partir
   d'un arbre de décision. Le but est de diviser des données en plusieurs étapes et en prenant à chaque fois
   le meilleurs attributs pour diviser le plus efficacement l'ensemble de données que nous avons.

  \section{Question de bon sens}
  
  \section{Jeu du pendu}
  Durant cet exercice, nous allons travailler avec une base de données contenant des noms d'animaux. Il nous
  faudra réaliser un algorithme qui va rechercher le nom d'un animal que l'utilisateur aura choisi.\\
  
  L'algorithme que nous développons utilise un arbre de décision afin d'optimiser la séléction d'attributs qui
  divisera les ensemble de nom. Pour commencer, nous créons un tableau dont chaque indice représente le numéro
  d'une lettre et qui stock le nombre de mot qui utilise une certaine lettre. Cela nous permet de calculer
  la probabilité qu'une lettre soit dans un mot avec le calcul suivant :
  $$ p = h / n $$
  n : nombre de mot\\
  h : tableau construit précédement\\
  p : probabilité qu'une lettre apparaisse dans un mot\\
  
\end{document}  