% --------------------------------------
% Document Class
% --------------------------------------
\documentclass{article}
% --------------------------------------



% --------------------------------------
% Use Package
% --------------------------------------

% french, english
\usepackage[francais]{babel}

% font, french accent
\usepackage[utf8]{inputenc} 
\usepackage[T1]{fontenc} 

% page layout
\usepackage{geometry}

% hypertext link
\usepackage[pdfpagelabels]{hyperref}

\usepackage{graphicx}
\usepackage{float}
\usepackage{verbatim}
\usepackage{fancyhdr}
\usepackage{amsmath}


% include pdf
\usepackage[final]{pdfpages}


% --------------------------------------



% --------------------------------------
% Page setting
% --------------------------------------
%\pagestyle{empty}
\setlength{\headheight}{15pt}

\setcounter{secnumdepth}{3}
\setcounter{tocdepth}{2}

\makeatletter
\@addtoreset{chapter}{part}
\makeatother 

\hypersetup{         % parametrage des hyperliens
  colorlinks=true,      % colorise les liens
  breaklinks=true,      % permet les retours à la ligne pour les liens trop longs
  urlcolor= blue,       % couleur des hyperliens
  linkcolor= black,     % couleur des liens internes aux documents (index, figures, tableaux, equations,...)
  citecolor= green      % couleur des liens vers les references bibliographiques
}

% --------------------------------------

% --------------------------------------
% Information
% --------------------------------------
\title{Compte-rendu TP2 Rdf : Codage d'un contour}
\author{Elliot VANEGUE et Gaëtan DEFLANDRE}
% --------------------------------------



% --------------------------------------
% Begin content
% --------------------------------------
\begin{document}

  % Set language to english
  \selectlanguage{francais}

  % Start the page counting
  \pagenumbering{arabic}

  \maketitle
  
  \mbox{}
  \newpage
  \clearpage
  
  \section{Introduction}
  
  \section{Code R}
  La variable cont est un vecteur.
  Le dernier parmaètre de plot permet de définir les limites de l'axe y. 
  \section{Descripteurs de Fourier}
  Z0 correspond au barycentre de la forme.
  % C(t) = X(t)+ i.Y(t)
  % TODO probleme avec l'addition de la constante complexe
  \section{Réduction d'une chaîne de contour}
  \section{Comparaison des deux approches}
  rect :
    fourier -> 1
    corde -> 0.5 -> best (peu de point)
    
  carre :
    fourier -> 1
    corde -> 0.5 -> best (peu de point)
    
  triangle :
    fourier -> 1
    corde -> 0.8
    
\end{document}